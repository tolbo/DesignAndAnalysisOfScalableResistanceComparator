\documentclass[twocolumn,conference]{IEEEtran}
\usepackage[T1]{fontenc}
\usepackage[latin9]{inputenc}
\usepackage{url}
\usepackage{graphicx, import, xcolor}
\usepackage{soul}
\usepackage{multirow}
\renewcommand\IEEEkeywordsname{Keywords}
\usepackage{amsmath}
\usepackage{cases}
\usepackage{makecell}
\usepackage[section]{placeins}
\usepackage{subcaption}
\usepackage{mathtools}
\DeclarePairedDelimiter\abs{\lvert}{\rvert}
\newcolumntype{C}[1]{>{\centering\let\newline\\\arraybackslash\hspace{0pt}}m{#1}}
\newcolumntype{R}[1]{>{\raggedleft\let\newline\\\arraybackslash\hspace{0pt}}m{#1}}
\makeatletter
\DeclareRobustCommand{\textsupsub}[2]{{%
  \m@th\ensuremath{%
    ^{\mbox{\fontsize\sf@size\z@#1}}%
    _{\mbox{\fontsize\sf@size\z@#2}}%
  }%
}}
\makeatother

\begin{document}
\title{Circuits to Access Resistive Memories with High State Capacity}
%\author{
%      Thanasin Bunnam,
%      Oleg Mayevsky,
%      Ahmed Soltan,
%      Danil Sokolov,
%      Alex Yakovlev\\
%      School of Electrical and Electronic Engineering, Newcastle University, UK
% }

\maketitle

\begin{abstract}
Metastability is a special consideration in many emerging memory technologies. This paper discusses trade-offs and optimizations that involve metastable behavior in a memristor-based memory. Following our prior experience in handling metastability in A-to-D converters, we propose a new method for designing a mechanism to read a memristor with high state capacity. The method uses an accurate sense amplifier with metastability resolution and completion detection, and shows robustness against PVT variations due to the use of reference resistor set with resistive buffers. Furthermore, the single supply voltage requirement is the outstanding feature among the available research works.  For demonstration of the proof of concept, a 2-bit memristor-based multi-bits memory cell (3MCell) has been built. It consists of four components: resistance comparator, reference resistor set, memristor and interface, and controllers which are built in synchronous and asynchronous versions. Simulation results support our extensive validation of the proposed circuits.

\end{abstract}

\begin{IEEEkeywords}
memristor, multi-bits memory, resistance comparator, metastability, memory controller
\end{IEEEkeywords}

\section{Introduction}
\label{sec:introduction}
\cite{Bunnam-2017-PATMOS}


\section{Conclusion}
\label{sec:Conclusion}


\bibliographystyle{IEEEtran}
\bibliography{refs}
\end{document}
